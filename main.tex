\documentclass[algorithmlist,figurelist,tablelist,nomlist,phd,UTF8]{seuthesix}

\usepackage{ctex}
\usepackage{booktabs}
\usepackage{amsthm}
\usepackage{amssymb}
\usepackage{url}
\usepackage{colortbl}
\usepackage{listings}

\usepackage{tcolorbox}
\tcbuselibrary{skins,breakable}

% \usepackage{amsmath}
% \usepackage{theorem}
% \newtheorem{theorem}{Theorem}
% \newcommand{\myblue}{\color[rgb]{0.21,0.49,0.74}}

% \usepackage{algorithm}
\usepackage{algorithmicx}
\usepackage{algpseudocode}
\usepackage{pgfplots}
\usepackage{tabularx}
\usepackage{rotating}

\usepackage{multirow}
\usepackage{array}

\pgfplotsset{compat=1.18}  % 使用最新版本,也可以用其他版本号

\definecolor{green2}{rgb}{0.21,0.74,0.49}
\lstset{
    language=Python,
    frame=single,
    backgroundcolor=\color{gray!20},
    basicstyle=\small\ttfamily,
    keywordstyle=\color{blue},
    commentstyle=\itshape\color{green2},
    numbers=left,
    stepnumber=1,
    numberstyle=\tiny
}

\tcbuselibrary{skins,breakable}
\newtcolorbox{myshadowbox}{
  enhanced,
  breakable,
  drop shadow southeast,
  sharp corners,
  colback=white,
  colframe=black
}
\newcolumntype{Y}{>{\centering\arraybackslash}X}
\definecolor{BlueViolet}{rgb}{0.21,0.49,0.74}

\newtheorem{theorem}{定理}
% \newtheorem{theorem}{Theorem}
\newcommand{\myblue}{\color[rgb]{0.21,0.49,0.74}}

% set main font
% \setCJKmainfont{FandolSong} 


\begin{document}
% ---------------------------------------- %
% % 设置PDF压缩级别 (0-9),9为最大压缩
% \pdfcompresslevel=9
% % 设置图片压缩质量
% \pdfimageresolution=150  % 设置图片分辨率为150dpi(默认是300dpi)
% ---------------------------------------- %

\categorynumber{TP391.4} % 分类采用《中国图书资料分类法》
\UDC{004.8}            %《国际十进分类法UDC》的类号
\secretlevel{公开}    %学位论文密级分为"公开"、"内部"、"秘密"和"机密"四种
\studentid{000000}   %学号要完整,前面的零不能省略。
\title{论文题目中文}{}{Thesis Title in English}{}
% ================================ %
\author{作者姓名}{Author Name}
\advisor{导师姓名}{教授}{Advisor Name}{Prof.}
\coadvisor{副导师姓名}{教授}{Co-advisor Name}{Prof.} % 没有可以不填
\degreetype{工学博士}{Doctor of Philosophy in Software Engineering} % 详细学位名称
\major{软件工程}
% ================================ %
\submajor{}
\defenddate{2025年XX月XX日}
\authorizedate{}
\committeechair{委员会主席姓名~教授}
\reviewer{评阅人}{}
\department{东南大学计算机科学与工程学院}{School of Computer Science and Engineering}
% \seuthesisthanks{本课题的研究获XXX基金资助}
\makebigcover
\makecover
\begin{abstract}{关键词1,关键词2,关键词3,关键词4,关键词5}
    [这里是中文摘要的内容。请根据您的研究内容填写详细的中文摘要,包括研究背景、研究目的、主要方法、关键结果和结论等。摘要应该简明扼要地概括您的整个研究工作,通常在300-500字之间。]

    针对上述挑战,本文对{\heiti [您的研究主题]}展开系统深入研究,主要研究内容和创新包括:

\begin{itemize}
\setlength{\itemsep}{0pt}
\item {\heiti [研究内容1]}:[详细描述第一个主要研究内容和创新点]

\item {\heiti [研究内容2]}:[详细描述第二个主要研究内容和创新点]

\item {\heiti [研究内容3]}:[详细描述第三个主要研究内容和创新点]

\item {\heiti [研究内容4]}:[详细描述第四个主要研究内容和创新点]
\end{itemize}

本文的系统研究不仅在理论上深化了对[相关领域]的理解,也在实践中为多个应用场景提供了可靠的技术解决方案。所提出的方法在[具体应用领域]等任务上均取得了显著进展,有效解决了[具体问题]。这些成果不仅促进了[相关理论]的发展,也为[应用领域]等领域提供了新的技术路径。
  
  \end{abstract}
  
  \begin{englishabstract}{Keyword1, Keyword2, Keyword3, Keyword4, Keyword5}
    [This is the English abstract content. Please fill in a detailed English abstract based on your research, including research background, objectives, main methods, key results, and conclusions. The abstract should concisely summarize your entire research work, typically between 200-400 words.]

Addressing these challenges, we present systematic research on {\heiti [Your Research Topic]}, establishing a comprehensive technical approach. The main research content and innovations include:

\begin{itemize}
\setlength{\itemsep}{0pt}
\item {\bfseries [Research Content 1]}: [Detailed description of the first main research content and innovation]

\item {\bfseries [Research Content 2]}: [Detailed description of the second main research content and innovation]

\item {\bfseries [Research Content 3]}: [Detailed description of the third main research content and innovation]

\item {\bfseries [Research Content 4]}: [Detailed description of the fourth main research content and innovation]
\end{itemize}

This systematic research not only deepens the theoretical understanding of [relevant field] but also provides reliable technical solutions for multiple application scenarios. The proposed methods have achieved significant progress in [specific application areas], effectively addressing [specific problems]. These achievements not only advance [relevant theory] but also provide new technical pathways for [application fields].
  
\end{englishabstract}

% \setnomname{术语与符号约定}
\tableofcontents
\listofothers

\mainmatter

% \nomenclature{PF}{powerful fingers}
% \nomenclature{KF}{kung fu}


\chapter{绪论}

\section{研究背景与意义}

% ==========================================================================
% 国内外研究现状
% ==========================================================================

\section{国内外研究现状}


% ==========================================================================
% 本文工作
% ==========================================================================


\section{本文工作}

\subsection{解决方案}


% 将下面的章节文件名和内容替换为您的实际章节
\chapter{[章节标题]}

\section{引言}

[在此处填写本章的引言部分,介绍本章要解决的问题、研究背景和主要贡献]
\cite{示例数据集2023}

\section{相关工作}

[在此处介绍与本章内容相关的已有研究工作,分析现有方法的优势和不足]

\section{方法}

[在此处详细介绍您提出的方法,包括:]

\subsection{问题定义}

[定义要解决的具体问题]

\subsection{方法概述}

[概述您的方法的整体思路和框架]

\subsection{具体算法}

[详细描述您的算法或方法的具体实现]

\section{实验}

[在此处描述实验设置和结果]

\subsection{实验设置}

[描述实验环境、数据集、评估指标等]

\subsection{实验结果}

[展示和分析实验结果]

\subsection{消融实验}

[如果有消融实验,在此处描述]

\section{本章小结}

[总结本章的主要内容和贡献]


% 如果有更多章节,请按以下格式添加:
% \input{chapters/chapter2.tex}
% \input{chapters/chapter3.tex}
% \input{chapters/chapter4.tex}

\chapter{总结与展望}

\section{研究工作总结}

本研究聚焦于[您的研究领域]在[具体应用领域]的关键挑战,特别是围绕[核心技术方法]展开深入探索。我们系统性地研究了从[研究角度1]到[研究角度2],再到[研究角度3]等多个层面的核心问题。具体而言,研究旨在[研究目标1],[研究目标2],[研究目标3],并将其拓展应用于[应用场景]等复杂场景。通过系统的理论分析与实验验证,本研究提出了一系列创新性解决方案,取得了显著的研究成果。

首先,针对[核心问题1],本研究创新性地引入了[方法论1]。我们提出了基于[理论基础]的[具体方法名称],有效解决了[具体问题],从而显著提升了[性能指标]。

在[第一个研究内容]的同时,本研究进一步深入探究了[核心机制],特别是[具体技术细节]的理论关系。我们构建了一个系统化的[技术框架名称],通过理论整合,揭示了[理论发现]。基于此理解,我们设计了新颖的[技术方案],其有效性在多个标准学术基准测试中得到了充分验证,为[技术发展方向]提供了理论指导。

基于对[前期研究内容]的深入理解,本研究着力于将[技术方法]的强大能力迁移至具体的[应用任务],并拓展其在[数据类型]上的应用。一方面,我们针对[具体任务1],提出了[方法1]的方法,实现了[效果描述]。另一方面,为了解决[技术挑战],特别是在[具体应用场景]这一复杂任务中,我们设计了[技术方案名称]。该[技术类型]通过引入[技术创新点]和优化的[技术组件],成功将[前期成果]的知识迁移并应用于[具体应用],展示了所提方法在处理[技术挑战]上的潜力。

综上所述,本研究通过解决[挑战1]、[挑战2]以及[挑战3]等多个层面的关键问题,为[研究领域]贡献了系统的创新性解决方案。研究不仅深化了对[理论内容]等技术的理论理解,也为提升其实际应用中的[性能指标1]、实现向[应用方向1]和[应用方向2]的有效拓展提供了经过验证的实用方法。

\section{未来工作展望}

[您的研究领域]在[应用领域]已经取得了广泛应用,同时也面临诸多技术挑战。本节从[展望维度1]、[展望维度2]、[展望维度3]、[展望维度4]以及[展望维度5]等五个维度进行展望。

{\heiti [展望方向1]}
尽管[当前技术]等方法取得了显著进展,其[技术限制]仍然是广泛应用的主要障碍。未来的研究将持续致力于[改进方向1]、[改进方向2]以及[改进方向3]。

在[技术层面1],虽然[当前主流技术]在[优势方面]表现出色,但其[技术问题]限制了其在[应用场景]的应用。因此,探索更[技术特性1]、更[技术特性2]的[技术变种]或全新的[技术类型],对于[技术目标]至关重要,尤其在处理[挑战性任务]这类[任务特点]的任务时。设计更[性能要求]的[技术组件]也是未来的重要方向。

在[技术层面2],多种技术值得进一步研究。例如,[技术方法1]通过在[应用阶段]引入[技术操作],使模型学习适应[技术要求],旨在[技术目标],并有潜力[额外收益]。[技术方法2]是另一种有效手段,可以利用[技术原理]来指导[技术目标]的训练。这不仅旨在[主要目标],更有可能帮助[次要目标],从而在保证[质量指标]的同时,显著降低[成本指标]。这些优化将直接推动[技术应用]在[应用场景]下的应用,并加速更[规模特征]模型的研发迭代。

{\heiti [展望方向2]}
随着[技术发展背景]以及[需求变化],实现[目标1]、[目标2]及[目标3]成为核心研究目标。目前主流[技术类型]在[权衡方面]存在权衡,如何在保证[质量指标]的前提下降低[成本指标],特别是[具体优化目标],已成为该领域的重要研究方向。

[对比技术1]凭借其[技术优势],在特定场景下表现出色。然而,[对比技术1]在[技术局限]方面存在明显局限性,往往难以[具体问题],导致[问题结果]。此外,[技术问题1]和[技术问题2]等问题进一步制约了其在[应用领域]中的应用。

相比之下,[对比技术2]通过[技术过程],在[复杂任务]等复杂任务中展现出卓越性能。然而,这类模型的[性能问题],通常需要[具体数量指标],导致[问题描述],严重限制了其在[应用场景]中的部署。

针对这一挑战,研究界已提出[解决思路数量]类主要优化思路。第一类基于[技术路线1],将[技术过程]建模为[数学模型],并通过[优化方法]优化[目标过程]。例如,[具体算法]已将[性能指标]降至[具体数值],但理论分析表明,在保证[质量要求]的前提下,[技术路线1]难以突破[理论极限]的理论下限。

第二类方法基于[技术路线2],如[具体方法1]和[具体方法2],通过[技术过程]将[源技术]知识压缩至[目标技术]中。最新研究表明,这类方法有潜力实现[性能目标],但当前仍面临[挑战1]和[挑战2]等挑战。

{\heiti [展望方向3]}
随着[技术发展]的快速发展,[技术系统]亟需提升[能力1]与[能力2]能力。在[具体挑战1]方面,模型需要[具体要求1]并生成[期望输出]。[具体挑战2]要求模型[技术要求],避免[问题描述]。

未来研究可聚焦于提升模型的[核心能力],如通过[技术手段]方法增强模型对[数据类型]的理解能力。同时,如何在[应用场景]中保持[质量指标],也是值得深入探索的研究方向。

总体而言,[研究领域]研究正处于快速发展阶段。未来工作需在[平衡维度1]、[平衡维度2]与[平衡维度3]间取得平衡,推动其在更广泛领域的应用落地与技术突破。本研究的相关成果为后续研究提供了有益的理论基础与技术参考。

% \acknowledgement
% 衷心感谢导师[导师姓名]教授对我的帮助与指导。

\thesisbib{seuthesix}


\appendix

\chapter{相关附录}


\resume{作者简介}

% \begin{flushleft}
% {\bfseries \large 作者简历}
% \end{flushleft}

[出生年月]出生于[出生地]。

[入学年月]考入东南大学[学院名称][专业名称],[毕业年月]本科毕业并获得[学位名称]。

[入学年月]免试进入东南大学计算机科学与工程学院[专业名称]攻读博士学位至今。

\begin{flushleft}
{\bfseries \large 攻读博士学位期间发表的论文}
\end{flushleft}

\noindent
(* 表示共同第一作者)

\noindent
[1] \textbf{作者姓名}, 合作者姓名1, 合作者姓名2, et al. 论文题目1. 会议/期刊名称 (年份): 页码. (期刊/会议级别)

\noindent
[2] \textbf{作者姓名}, 合作者姓名1, 合作者姓名2, et al. 论文题目2. 会议/期刊名称 (年份): 页码. (期刊/会议级别)

\noindent
[3] \textbf{作者姓名}, 合作者姓名1, 合作者姓名2, et al. 论文题目3. 会议/期刊名称 (年份): 页码. (期刊/会议级别)

\noindent
[添加更多论文...]


\begin{flushleft}
  {\bfseries \large 投稿论文}
\end{flushleft}

\noindent

\noindent
[1] 作者姓名*, \textbf{共同作者姓名}*, 其他作者姓名, et al. 投稿论文题目1. arXiv preprint arXiv:xxxx.xxxxx. 投稿至[会议/期刊名称].

\noindent
[2] 作者姓名, \textbf{共同作者姓名}, 其他作者姓名. 投稿论文题目2. 投稿至[会议/期刊名称].


% \board{毕业/学位论文答辩委员会名单}

\begin{table}[h]
    \centering
    % \renewcommand{\arraystretch}{1.5}
    \begin{tabular}{|c|c|c|c|c|}
    \hline
    \multicolumn{1}{|c|}{\textbf{毕业/学位论文题目}} & \multicolumn{4}{|c|}{[论文题目]} \\
    \hline
    \multicolumn{1}{|c|}{\textbf{作者}} & \multicolumn{4}{|c|}{[作者姓名]} \\
    \hline
    \multicolumn{1}{|c|}{\textbf{专业}} & \multicolumn{4}{|c|}{[专业名称]} \\
    \hline
    \multicolumn{1}{|c|}{\textbf{研究方向}} & \multicolumn{4}{|c|}{[研究方向]} \\
    \hline
    \multicolumn{1}{|c|}{\textbf{导师}} & \multicolumn{4}{|c|}{[导师姓名]} \\
    \hline
    \multirow{4}{*}{\begin{tabular}[c]{@{}c@{}}答\\辩\\委\\员\\会\\组\\成\end{tabular}}
    & \textbf{姓名} & \textbf{职称} & \textbf{学科专业} & \textbf{工作单位} \\
    \cline{2-5}
    & [委员姓名]\textbf{(主席)} & [职称] & [学科专业] & [工作单位] \\
    \cline{2-5}
    & [委员姓名] & [职称] & [学科专业] & [工作单位] \\
    \cline{2-5}
    & [委员姓名] & [职称] & [学科专业] & [工作单位] \\
    \cline{2-5}
    & [委员姓名] & [职称] & [学科专业] & [工作单位] \\
    \cline{2-5}
    & [委员姓名]\textbf{(秘书)} & [职称] & [学科专业] & [工作单位] \\
    \hline
    \end{tabular}
\end{table}

\end{document}